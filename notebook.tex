



    
\documentclass[11pt]{article}
    
    \usepackage{parskip}
    \setcounter{secnumdepth}{0} %Suppress section numbers
    \usepackage[breakable]{tcolorbox}
    \tcbset{nobeforeafter}
    \usepackage{needspace}
    \usepackage{minted}
    \usemintedstyle{jupyter_python}
    
    \usepackage[T1]{fontenc}
    % Nicer default font (+ math font) than Computer Modern for most use cases
    \usepackage{mathpazo}

    % Basic figure setup, for now with no caption control since it's done
    % automatically by Pandoc (which extracts ![](path) syntax from Markdown).
    \usepackage{graphicx}
    % We will generate all images so they have a width \maxwidth. This means
    % that they will get their normal width if they fit onto the page, but
    % are scaled down if they would overflow the margins.
    \makeatletter
    \def\maxwidth{\ifdim\Gin@nat@width>\linewidth\linewidth
    \else\Gin@nat@width\fi}
    \makeatother
    \let\Oldincludegraphics\includegraphics
    % Set max figure width to be 80% of text width, for now hardcoded.
    \renewcommand{\includegraphics}[1]{\Oldincludegraphics[width=.8\maxwidth]{#1}}
    % Ensure that by default, figures have no caption (until we provide a
    % proper Figure object with a Caption API and a way to capture that
    % in the conversion process - todo).
    \usepackage{caption}
    \DeclareCaptionLabelFormat{nolabel}{}
    \captionsetup{labelformat=nolabel}

    \usepackage{adjustbox} % Used to constrain images to a maximum size 
    \usepackage{xcolor} % Allow colors to be defined
    \usepackage{enumerate} % Needed for markdown enumerations to work
    \usepackage{geometry} % Used to adjust the document margins
    \usepackage{amsmath} % Equations
    \usepackage{amssymb} % Equations
    \usepackage{textcomp} % defines textquotesingle
    % Hack from http://tex.stackexchange.com/a/47451/13684:
    \AtBeginDocument{%
        \def\PYZsq{\textquotesingle}% Upright quotes in Pygmentized code
    }
    \usepackage{upquote} % Upright quotes for verbatim code
    \usepackage{eurosym} % defines \euro
    \usepackage[mathletters]{ucs} % Extended unicode (utf-8) support
    \usepackage[utf8x]{inputenc} % Allow utf-8 characters in the tex document
    \usepackage{fancyvrb} % verbatim replacement that allows latex
    \usepackage{grffile} % extends the file name processing of package graphics 
                         % to support a larger range 
    % The hyperref package gives us a pdf with properly built
    % internal navigation ('pdf bookmarks' for the table of contents,
    % internal cross-reference links, web links for URLs, etc.)
    \usepackage{hyperref}
    \usepackage{longtable} % longtable support required by pandoc >1.10
    \usepackage{booktabs}  % table support for pandoc > 1.12.2
    \usepackage[inline]{enumitem} % IRkernel/repr support (it uses the enumerate* environment)
    \usepackage[normalem]{ulem} % ulem is needed to support strikethroughs (\sout)
                                % normalem makes italics be italics, not underlines
    

    \let\Oldtex\TeX     % provide compatibility with nbconvert <= 5.3.1
    \let\Oldlatex\LaTeX % pre-included in nbconvert > 5.3.1 so redundant
    
    % Colors for the hyperref package
    \definecolor{urlcolor}{rgb}{0,.145,.698}
    \definecolor{linkcolor}{rgb}{.71,0.21,0.01}
    \definecolor{citecolor}{rgb}{.12,.54,.11}

    % ANSI colors
    \definecolor{ansi-black}{HTML}{3E424D}
    \definecolor{ansi-black-intense}{HTML}{282C36}
    \definecolor{ansi-red}{HTML}{E75C58}
    \definecolor{ansi-red-intense}{HTML}{B22B31}
    \definecolor{ansi-green}{HTML}{00A250}
    \definecolor{ansi-green-intense}{HTML}{007427}
    \definecolor{ansi-yellow}{HTML}{DDB62B}
    \definecolor{ansi-yellow-intense}{HTML}{B27D12}
    \definecolor{ansi-blue}{HTML}{208FFB}
    \definecolor{ansi-blue-intense}{HTML}{0065CA}
    \definecolor{ansi-magenta}{HTML}{D160C4}
    \definecolor{ansi-magenta-intense}{HTML}{A03196}
    \definecolor{ansi-cyan}{HTML}{60C6C8}
    \definecolor{ansi-cyan-intense}{HTML}{258F8F}
    \definecolor{ansi-white}{HTML}{C5C1B4}
    \definecolor{ansi-white-intense}{HTML}{A1A6B2}

    % commands and environments needed by pandoc snippets
    % extracted from the output of `pandoc -s`
    \providecommand{\tightlist}{%
      \setlength{\itemsep}{0pt}\setlength{\parskip}{0pt}}
    \DefineVerbatimEnvironment{Highlighting}{Verbatim}{commandchars=\\\{\}}
    % Add ',fontsize=\small' for more characters per line
    \newenvironment{Shaded}{}{}
    \newcommand{\KeywordTok}[1]{\textcolor[rgb]{0.00,0.44,0.13}{\textbf{{#1}}}}
    \newcommand{\DataTypeTok}[1]{\textcolor[rgb]{0.56,0.13,0.00}{{#1}}}
    \newcommand{\DecValTok}[1]{\textcolor[rgb]{0.25,0.63,0.44}{{#1}}}
    \newcommand{\BaseNTok}[1]{\textcolor[rgb]{0.25,0.63,0.44}{{#1}}}
    \newcommand{\FloatTok}[1]{\textcolor[rgb]{0.25,0.63,0.44}{{#1}}}
    \newcommand{\CharTok}[1]{\textcolor[rgb]{0.25,0.44,0.63}{{#1}}}
    \newcommand{\StringTok}[1]{\textcolor[rgb]{0.25,0.44,0.63}{{#1}}}
    \newcommand{\CommentTok}[1]{\textcolor[rgb]{0.38,0.63,0.69}{\textit{{#1}}}}
    \newcommand{\OtherTok}[1]{\textcolor[rgb]{0.00,0.44,0.13}{{#1}}}
    \newcommand{\AlertTok}[1]{\textcolor[rgb]{1.00,0.00,0.00}{\textbf{{#1}}}}
    \newcommand{\FunctionTok}[1]{\textcolor[rgb]{0.02,0.16,0.49}{{#1}}}
    \newcommand{\RegionMarkerTok}[1]{{#1}}
    \newcommand{\ErrorTok}[1]{\textcolor[rgb]{1.00,0.00,0.00}{\textbf{{#1}}}}
    \newcommand{\NormalTok}[1]{{#1}}
    
    % Additional commands for more recent versions of Pandoc
    \newcommand{\ConstantTok}[1]{\textcolor[rgb]{0.53,0.00,0.00}{{#1}}}
    \newcommand{\SpecialCharTok}[1]{\textcolor[rgb]{0.25,0.44,0.63}{{#1}}}
    \newcommand{\VerbatimStringTok}[1]{\textcolor[rgb]{0.25,0.44,0.63}{{#1}}}
    \newcommand{\SpecialStringTok}[1]{\textcolor[rgb]{0.73,0.40,0.53}{{#1}}}
    \newcommand{\ImportTok}[1]{{#1}}
    \newcommand{\DocumentationTok}[1]{\textcolor[rgb]{0.73,0.13,0.13}{\textit{{#1}}}}
    \newcommand{\AnnotationTok}[1]{\textcolor[rgb]{0.38,0.63,0.69}{\textbf{\textit{{#1}}}}}
    \newcommand{\CommentVarTok}[1]{\textcolor[rgb]{0.38,0.63,0.69}{\textbf{\textit{{#1}}}}}
    \newcommand{\VariableTok}[1]{\textcolor[rgb]{0.10,0.09,0.49}{{#1}}}
    \newcommand{\ControlFlowTok}[1]{\textcolor[rgb]{0.00,0.44,0.13}{\textbf{{#1}}}}
    \newcommand{\OperatorTok}[1]{\textcolor[rgb]{0.40,0.40,0.40}{{#1}}}
    \newcommand{\BuiltInTok}[1]{{#1}}
    \newcommand{\ExtensionTok}[1]{{#1}}
    \newcommand{\PreprocessorTok}[1]{\textcolor[rgb]{0.74,0.48,0.00}{{#1}}}
    \newcommand{\AttributeTok}[1]{\textcolor[rgb]{0.49,0.56,0.16}{{#1}}}
    \newcommand{\InformationTok}[1]{\textcolor[rgb]{0.38,0.63,0.69}{\textbf{\textit{{#1}}}}}
    \newcommand{\WarningTok}[1]{\textcolor[rgb]{0.38,0.63,0.69}{\textbf{\textit{{#1}}}}}
    
    
    % Define a nice break command that doesn't care if a line doesn't already
    % exist.
    \def\br{\hspace*{\fill} \\* }
    % Math Jax compatability definitions
    \def\gt{>}
    \def\lt{<}
    % Document parameters
    \title{Assignment 2}
    
    
    
% Pygments definitions
    
    \makeatletter
    \newcommand*\@iflatexlater{\@ifl@t@r\fmtversion}
    \@iflatexlater{2016/03/01}{
	    \newcommand{\wordboundary}{4095}}{
	    \newcommand{\wordboundary}{255}}
    \makeatother

    \newif\ifcode
    \codefalse
    \definecolor{Grey}{rgb}{0.40,0.40,0.40}
    %If using XeLaTeX, use magic to not highlight . operators with purple.
    \ifdefined\XeTeXcharclass
    \XeTeXinterchartokenstate = 1
    \newXeTeXintercharclass \mycharclassGrey
    \XeTeXcharclass `. \mycharclassGrey
    \XeTeXinterchartoks 0 \mycharclassGrey   = {\bgroup\ifcode\color{Grey}\else\fi}

    \XeTeXinterchartoks \wordboundary \mycharclassGrey = {\bgroup\ifcode\color{Grey}\else\fi}

    \XeTeXinterchartoks \mycharclassGrey 0   = {\egroup}
    \XeTeXinterchartoks \mycharclassGrey \wordboundary = {\egroup}
    \fi %end magical operator highlighting
    %End Reconfigured Pygments
    
   
    % Exact colors from NB
    \definecolor{incolor}{HTML}{303F9F}
    \definecolor{outcolor}{HTML}{D84315}
    \definecolor{cellborder}{HTML}{CFCFCF}
    \definecolor{cellbackground}{HTML}{F7F7F7}

    % needed definitions
    \newif\ifleftmargins
    \newlength{\promptlength}

    % cell style settings
        \leftmarginsfalse

    
    % prompt
    \newcommand{\prompt}[3]{
        \needspace{1.1cm}
        \settowidth{\promptlength}{ #1 [#3] }
        \ifleftmargins\hspace{-\promptlength}\hspace{-5pt}\fi
        {\color{#2}#1 [#3]:}
        \ifleftmargins\vspace{-2.7ex}\fi
    }
    
    
    % environments
    \newenvironment{OutVerbatim}{\VerbatimEnvironment%
        \begin{tcolorbox}[breakable, boxrule=.5pt, size=fbox, pad at break*=1mm, opacityfill=0]
            \begin{Verbatim}
            }{
            \end{Verbatim}
        \end{tcolorbox}
    }
    
    %Updated MathJax Compatibility (if future behaviour of the notebook changes this may be removed)
    \renewcommand{\TeX}{\ifmmode \textrm{\Oldtex} \else \textbackslash TeX \fi}
    \renewcommand{\LaTeX}{\ifmmode \Oldlatex \else \textbackslash LaTeX \fi}
    
    % Header Adjustments
    \renewcommand{\paragraph}{\textbf}
    \renewcommand{\subparagraph}[1]{\textit{\textbf{#1}}}

    
    % Prevent overflowing lines due to hard-to-break entities
    \sloppy 
    % Setup hyperref package
    \hypersetup{
      breaklinks=true,  % so long urls are correctly broken across lines
      colorlinks=true,
      urlcolor=urlcolor,
      linkcolor=linkcolor,
      citecolor=citecolor,
      }
    % Slightly bigger margins than the latex defaults
    \geometry{verbose,tmargin=.5in,bmargin=.7in,lmargin=.5in,rmargin=.5in}
    

    \begin{document}
    
    
    
    
    

    
    \hypertarget{assignment-2}{%
\section{Assignment 2}\label{assignment-2}}

    
\prompt{In}{incolor}{1}
\codetrue
\begin{tcolorbox}[breakable, size=fbox, boxrule=1pt, pad at break*=1mm, colback=cellbackground, colframe=cellborder]
\begin{minted}[breaklines=True]{ipython3}
import numpy as np
import scipy as sp
import matplotlib as mp
import matplotlib.pyplot as plt

from scipy.integrate import quad, dblquad
from textwrap import wrap

%matplotlib inline
%config InlineBackend.figure_format = 'pdf'
\end{minted}
\end{tcolorbox}
\codefalse

    \hypertarget{question-5}{%
\subsection{Question 5}\label{question-5}}

    \hypertarget{a}{%
\subsubsection{(a)}\label{a}}

We have a loop of radius \(10\)cm lying in the xy-plane with current of
\(I\) flowing.

To calculate the magnetic field produced by this loop we use the
Biot-Savart Law:

\[\vec{B}(\vec{r}) = \frac{\mu_0 I}{4 \pi} \int \frac{\text{d}l \times \hat{r'}}{r'^2}\]

\[\text{where} \left\{ \begin{array}{ll}
            \vec{r}  & \text{is the position vector of the point of evaluation} \\
            \vec{r'} & \text{is the displacement vector from a point in the current loop to the point of evaluation} \\
            l        & \text{is the length of the current loop} \\
            \end{array} \right.\]

    \textbf{Let:}

\begin{quote}
Radius of the current loop: \(R\)
\end{quote}

\begin{quote}
Positions of elements on the loop: \(\vec{R}\)
\end{quote}

\begin{quote}
Angle from the center of the current loop to an element: \(\theta\)
\end{quote}

\begin{quote}
Line segment of the current loop: \(l\)
\end{quote}

\begin{quote}
Locations in space: \(\vec{r} = (x, y, z)\)
\end{quote}

    \textbf{Then:} \[\vec{R} = [R\cos(\theta),\ y-R\sin(\theta), \ z]\]

\[\text{d}l = R \ \text{d} \theta \ \hat{\theta} = [-R \sin(\theta), \ R \cos(\theta), \ 0] \ \text{d}\theta\]

\[\vec{r'} = \vec{r} - \vec{R} = [x-R\cos(\theta), \ y-R\sin(\theta), \ z]\]

\textbf{Taking the cross product:} \begin{align}
\text{d}l \times \vec{r'} &= zR\cos(\theta) \ \hat{x} + zR\sin(\theta) \ \hat{y} + [-R\sin(\theta)(y-R\sin(\theta)) - R\cos(\theta)(x-R\cos(\theta))] \ \hat{z} \ \text{d}\theta\\
&= zR\cos(\theta) \ \hat{x} + zR\sin(\theta) \ \hat{y} + [-yR\sin(\theta)-xR\cos(\theta) + R^2] \hat{z} \ \text{d}\theta 
\end{align}

\textbf{The term inside the intergral:} \begin{align}
\frac{\text{d}l \times \vec{r'}}{r'^2} &= \frac{zR\cos(\theta) \ \hat{x} + zR\sin(\theta) \ \hat{y} + [-yR\sin(\theta)-xR\cos(\theta) + R^2] \hat{z}}{\left\{[x-R\cos(\theta)]^2 + [y-R\sin(\theta)]^2 + z^2 \right\}^\frac{3}{2}}
\end{align}

\textbf{The magnetic field:}
\[\vec{B}(\vec{r}) = \frac{\mu_0 I}{4 \pi} \int \text{d}\theta \ \frac{zR\cos(\theta) \ \hat{x} + zR\sin(\theta) \ \hat{y} + [-yR\sin(\theta)-xR\cos(\theta) + R^2] \hat{z}}{\left\{[x-R\cos(\theta)]^2 + [y-R\sin(\theta)]^2 + z^2 \right\}^\frac{3}{2} }\]

\textbf{Make it dimensionless:}
\[\vec{B}(\vec{r}) \frac{4 \pi}{\mu_0 I}  = \int \text{d}\theta \ \frac{\frac{z}{R}\cos(\theta) \ \hat{x} + \frac{z}{R}\sin(\theta) \ \hat{y} - \left[\frac{y}{R}\sin(\theta) + \frac{x}{R}\cos(\theta) - 1\right] \hat{z}}{\left\{ \left[\frac{x}{R}-\cos(\theta)\right]^2 + \left[\frac{y}{R}-\sin(\theta)\right]^2 + \left(\frac{z}{R}\right)^2 \right\}^\frac{3}{2} }\]

    \textbf{Turn the magnetic field into code:}

    
\prompt{In}{incolor}{2}
\codetrue
\begin{tcolorbox}[breakable, size=fbox, boxrule=1pt, pad at break*=1mm, colback=cellbackground, colframe=cellborder]
\begin{minted}[breaklines=True]{ipython3}
# Magnetic fields in x
def integrand_x(x, y, z, theta, R=1):
    r = ((x/R - np.cos(theta))**2 + (y/R - np.sin(theta))**2 + (z/R)**2)**.5
    Bx = (z/R)*np.cos(theta) / r**3
    return Bx

@np.vectorize
def Bx(x, y, z, R=1):
    B_x = quad(lambda theta: integrand_x(x, y, z, theta, R), 0, 2*np.pi)
    return B_x[0]

# Magnetic fields in y
def integrand_y(x, y, z, theta, R=1):
    r = ((x/R - np.cos(theta))**2 + (y/R - np.sin(theta))**2 + (z/R)**2)**.5
    By = (z/R)*np.sin(theta) / r**3
    return By

@np.vectorize
def By(x, y, z, R=1):
    B_y = quad(lambda theta: integrand_y(x, y, z, theta, R), 0, 2*np.pi)
    return B_y[0]

# Magnetic fields in z
def integrand_z(x, y, z, theta, R=1):
    r = ((x/R - np.cos(theta))**2 + (y/R - np.sin(theta))**2 + (z/R)**2)**.5
    Bz = -( (y/R)*np.sin(theta) + (x/R)*np.cos(theta) - 1) / r**3
    return Bz

@np.vectorize
def Bz(x, y, z, R=1):
    B_z = quad(lambda theta: integrand_z(x, y, z, theta, R), 0, 2*np.pi)
    return B_z[0]
\end{minted}
\end{tcolorbox}
\codefalse

    
\prompt{In}{incolor}{3}
\codetrue
\begin{tcolorbox}[breakable, size=fbox, boxrule=1pt, pad at break*=1mm, colback=cellbackground, colframe=cellborder]
\begin{minted}[breaklines=True]{ipython3}
# Create a meshgrid for the space of integration
# ==============================================

# Dimesions and resolution for the meshgrid 
r_max = 2
r_min = -2
res = 200

x = np.linspace(r_min, r_max, res)
y = np.linspace(r_min, r_max, res)
z = np.linspace(r_min, r_max, res)

GX, GY, GZ = np.meshgrid(x, y, z)
Gx, Gz = np.meshgrid(x, z)
Gy = y
\end{minted}
\end{tcolorbox}
\codefalse

    
\prompt{In}{incolor}{4}
\codetrue
\begin{tcolorbox}[breakable, size=fbox, boxrule=1pt, pad at break*=1mm, colback=cellbackground, colframe=cellborder]
\begin{minted}[breaklines=True]{ipython3}
%%time

# Find the magnetic fields
# ========================

B_x = Bx(Gx, Gy, Gz, 1)
B_y = By(Gx, Gy, Gz, 1)
B_z = Bz(Gx, Gy, Gz, 1)
\end{minted}
\end{tcolorbox}
\codefalse

    \begin{Verbatim}[commandchars=\\\{\}]
CPU times: user 1min 4s, sys: 49.6 ms, total: 1min 4s
Wall time: 1min 4s

    \end{Verbatim}

    
\prompt{In}{incolor}{5}
\codetrue
\begin{tcolorbox}[breakable, size=fbox, boxrule=1pt, pad at break*=1mm, colback=cellbackground, colframe=cellborder]
\begin{minted}[breaklines=True]{ipython3}
# Plot it out
# ===========

fig, ax = plt.subplots(figsize=(7, 5))

norm = (B_x**2 + B_y**2 + B_z**2)**0.5

Graph = ax.pcolormesh(x, y, norm, cmap='inferno_r', vmax=25)
ax.streamplot(x, z, B_x, B_z, linewidth=0.75,
             density=1.5, arrowstyle='->', arrowsize=1)

# ax.quiver(x, z, B_x/norm, B_z/norm, linewidth=0.75, pivot='mid')
# Graph = ax.imshow(E[2], extent=(-2, 2, -2, 2), vmax=100, cmap='inferno_r')

ax.set_xlim(-2, 2)
ax.set_ylim(-2, 2)
ax.set_xlabel(r'Distance ($\hat{x}$) [dm]', fontsize=12)
ax.set_ylabel(r'Distance ($\hat{z}$) [dm]', fontsize=12)
ax.set_title('Magnetic field of a current loop in the xz-plane', fontsize=14)

cbar = fig.colorbar(Graph, shrink=1, aspect=20)
cbar.ax.set_ylabel(r'$\left| \vec{B} \right|$ $\left[\frac{4 \pi}{\mu_0 I}\right]$', 
                  rotation=0, fontsize=16, labelpad=40)

caption = r"Magnetic field of a current loop located in the xy-plane traveling counterclockwise. The heat map of field magnitude has a max cutoff value at $25 \frac{4 \pi}{\mu_0 I}$ for display purposes, but higher values exist. "
fig.text(0.05, -0.15, "\n".join(wrap(caption, 75)), ha='left', fontsize=12, wrap=False)

plt.show()
\end{minted}
\end{tcolorbox}
\codefalse

    \begin{center}
    \adjustimage{max size={0.9\linewidth}{0.9\paperheight}}{output_10_0.pdf}
    \end{center}
    { \hspace*{\fill} \\}
    
    \newpage

\hypertarget{b}{%
\subsubsection{(b)}\label{b}}

Please see attached hand written page for diagram

Here we have a small rotating loop of radius \(5\)cm located \(12\)cm
away from the original loop on the \(\hat{z}\) axis.

\textbf{Let:}

\begin{quote}
Radius of the small loop: \(s\)
\end{quote}

\begin{quote}
Position of elements in the small loop: \(\vec{S}\)
\end{quote}

\begin{quote}
Distance of the small loop from the original loop: \(z_0 \ \hat{z}\)
\end{quote}

\begin{quote}
Rate at which the small loop is rotating: \(\omega t \ \hat{x}\)
\end{quote}

\begin{quote}
Radial unit vector in the small loop: \(\hat{s}\)
\end{quote}

\begin{quote}
Azimuthal angle between the \(x\)-axis and \(\hat{s}\): \(\phi\)
\end{quote}

\begin{quote}
Normal vector to the surface enclosed by the small current loop:
\(\hat{n}\)
\end{quote}

\begin{quote}
Polar angle between \(\hat{n}\) and the \(z\)-axis: \(\varphi\)
\end{quote}

\begin{quote}
Displacement vector from an element in the original loop to an element
in the small loop: \(\vec{r''}\)
\end{quote}

From inspecting the diagram, we can see that:

\begin{quote}
\(\hat{s} = \cos{\phi}\hat{x} + \sin{\phi}\sin{\varphi} \hat{y} + \sin{\phi}\cos{\varphi} \hat{z}\)
\end{quote}

\begin{quote}
\(\hat{n} = \sin{\varphi}\hat{y} + \cos{\varphi}\hat{z}\)
\end{quote}

\begin{quote}
\(\vec{S} = S\cos{\phi}\hat{x} + S\sin{\phi}\sin{\varphi} \hat{y} + [S\sin{\phi}\cos{\varphi} + z_0] \hat{z}\)
\end{quote}

\begin{quote}
\(\vec{r''} = \vec{S} - \vec{R} = [S\cos{\phi} - R\cos{\theta}, \ S\sin{\phi}\sin{\varphi} - R\sin{\theta},\ S\sin{\phi}\cos{\varphi} + z_0]\)
\end{quote}

Since the loop is rotating along the \(\hat{x}\), we can conclude:

\begin{quote}
\(\varphi = \omega t\)
\end{quote}

    \textbf{Current induced in the loop:}

\begin{align}
I &= \int_0^{2\pi} \int_0^s \vec{B} \cdot \text{d} \vec{a} \\
  &= \int_0^{2\pi} \int_0^s \vec{B} \cdot \hat{n} \ s \ \text{d}s \ \text{d} \phi
\end{align}

    Code the current:

    
\prompt{In}{incolor}{6}
\codetrue
\begin{tcolorbox}[breakable, size=fbox, boxrule=1pt, pad at break*=1mm, colback=cellbackground, colframe=cellborder]
\begin{minted}[breaklines=True]{ipython3}
@np.vectorize
def B_norm(S, phi, Vphi, z_0=1.2):
    x = S*np.cos(phi)
    y = S*np.sin(phi)*np.sin(Vphi)
    z = S*np.sin(phi)*np.cos(Vphi) + z_0
    
    B = np.array([0, By(x, y, z), Bz(x, y, z)])
    n = np.array([0, np.sin(Vphi), np.cos(Vphi)])
    
    Bn = np.dot(B, n)
    return Bn

@np.vectorize
def current(R=1, s=0.5, z_0=1.2, Vphi=0.1*np.pi):
    I = dblquad(lambda phi, S: B_norm(S, phi, Vphi)*S, 0, s, lambda phi: 0, lambda phi: 2*np.pi)
    return I
\end{minted}
\end{tcolorbox}
\codefalse

    
\prompt{In}{incolor}{7}
\codetrue
\begin{tcolorbox}[breakable, size=fbox, boxrule=1pt, pad at break*=1mm, colback=cellbackground, colframe=cellborder]
\begin{minted}[breaklines=True]{ipython3}
VPhi = np.linspace(0, 2*np.pi, 100) 
\end{minted}
\end{tcolorbox}
\codefalse

    
\prompt{In}{incolor}{8}
\codetrue
\begin{tcolorbox}[breakable, size=fbox, boxrule=1pt, pad at break*=1mm, colback=cellbackground, colframe=cellborder]
\begin{minted}[breaklines=True]{ipython3}
%%time

I = current(Vphi = VPhi)
\end{minted}
\end{tcolorbox}
\codefalse

    \begin{Verbatim}[commandchars=\\\{\}]
CPU times: user 5min 24s, sys: 1.21 s, total: 5min 25s
Wall time: 5min 23s

    \end{Verbatim}

    
\prompt{In}{incolor}{10}
\codetrue
\begin{tcolorbox}[breakable, size=fbox, boxrule=1pt, pad at break*=1mm, colback=cellbackground, colframe=cellborder]
\begin{minted}[breaklines=True]{ipython3}
fig2, ax2 = plt.subplots(1, 1, figsize=(8,6))

ax2.plot(VPhi/(2*np.pi), I[0], label='Induced current', zorder=2)
ax2.plot(VPhi/(2*np.pi), np.max(I)*np.cos(VPhi), label='Cosine wave', linestyle='--', zorder=1)
ax2.grid()

ax2.set_xlabel(r'Time [s]', fontsize=14)
ax2.set_ylabel(r'Induced current $(I_{ind})$ $\left[\frac{4 \pi}{\mu_0 I s^2}\right]$', fontsize=14)
title2 = r'Induced current in the small loop above the current loop rotating at $2 \pi$ rad/s'
ax2.set_title("\n".join(wrap(title2, 60)), fontsize=16)

plt.show()
\end{minted}
\end{tcolorbox}
\codefalse

    \begin{center}
    \adjustimage{max size={0.9\linewidth}{0.9\paperheight}}{output_17_0.pdf}
    \end{center}
    { \hspace*{\fill} \\}
    

    % Add a bibliography block to the postdoc
    
    
    
    \end{document}
